\documentclass[a4paper,twoside,11pt]{article}
\usepackage[english]{babel}
\usepackage[T1]{fontenc}
\usepackage[ansinew]{inputenc}
\usepackage{geometry}
\usepackage{color} 
\geometry{a4paper,left=20mm,right=20mm, top=20mm, bottom=20mm} 

\usepackage{lmodern}

\usepackage{graphicx}

\usepackage{amsmath} \numberwithin{equation}{section}
\usepackage{amsthm}
\usepackage{amsfonts}
\usepackage{amssymb}
\usepackage{graphics}
\usepackage{color}

% theorems
\newtheorem{satz}{Theorem}
\renewcommand{\thesatz}{\Alph{satz}}
\theoremstyle{plain}
\newtheorem{theorem}{Theorem}[section]
\newtheorem{lemma}[theorem]{Lemma}
\newtheorem{corollary}[theorem]{Corollary}
\newtheorem{question}[theorem]{Question} 
\newtheorem{remark}[theorem]{Remark}
\theoremstyle{definition}
\newtheorem{definition}[theorem]{Definition}
\newtheorem{proposition}[theorem]{Proposition}    
\theoremstyle{remark}
\newtheorem{example}[theorem]{Example}

\definecolor{orange}{rgb}{1,0.5,0}

\usepackage{enumerate}
% proofs
\newcommand\proofsymbol{\frame{\rule[0pt]{0pt}{8pt}\rule[0pt]{8pt}{0pt}}}

% math
\newcommand\with{\ \vrule\ }  % 'mit'-Symbol in Mengen
\newcommand\abs[1]{\left|#1\right|} % Absolutbetrag
\newcommand\C{\mathbb C}         % K�rper der komplexen Zahlen
\newcommand\R{\mathbb R}         % K�rper der reellen Zahlen
\newcommand\Z{\mathbb Z}         % Ring der ganzen Zahlen
\newcommand\N{\mathbb N}         % Ring der ganzen Zahlen
\newcommand\Ha{\mathbb H}       
\newcommand\D{\mathbb D}     

\renewcommand\Im{\text{Im}}        
\renewcommand\Re{\text{Re}}       

\newenvironment{conjecture}[1][]{\begin{trivlist}
\item[\hskip \labelsep {\bfseries #1}]}{\end{trivlist}}

% other
      \newcommand{\eps}{\varepsilon}

\newcommand{\LandauO}{\mathcal{O}} 
\newcommand{\Landauo}{{\scriptstyle\mathcal{O}}}
\newcommand{\diam}{\operatorname{diam}}
\newcommand{\dist}{\operatorname{dist}}
\newcommand{\B}{\mathbb B}

\newcommand{\supp}{\operatorname{supp}} 

\usepackage[pdfstartview=FitH]{hyperref}


\begin{document}
\parindent 0pt 

\setcounter{section}{0}

%\title{Questions to: Multiple SLE and the complex Burgers equation}
%\date{\today}
%\maketitle

%\abstract{In this note we ask whether one can take the limit of multiple SLE as the number of slits goes to infinity. In the special case of $n$ slits that connect $n$ points of the boundary to one fixed point, one can take the limit of the Loewner equation that describes the growth of those slits in a simultaneous way. In this case, the limit is a deterministic Loewner equation whose vector field is determined by a complex Burgers equation.}\\

%{\bf Keywords:} stochastic Loewner evolution, multiple SLE, McKean-Vlasov equation, complex Burgers equation\\

%{\bf 2010 Mathematics Subject Classification:} 20M20, 32A40, 37L05.

\tableofcontents
\parindent 0pt


\section{The problem}

Let $N\in\N$ and $x_{N,1},...,x_{N,N}$ be $N$ points on $\partial\D$. Furthermore, choose $\lambda_{N,1},...,\lambda_{N,N}\in(0,1)$ such that $\sum_{k=1}^N\lambda_{N,k}=1.$\\

We define $N$ random processes $V_{N,1},...,V_{N,N}$ on $\R$ as the solution of the SDE system
\begin{equation}\label{sigma}
 dV_{N,k}(t) = \sum_{j\not=k}V_{N,k}(t)(\lambda_{N,k}+\lambda_{N,j})\frac{V_{N,j}(t)+V_{N,k}(t)}{V_{N,j}(t)-V_{N,k}(t)}dt-\frac{\kappa\lambda_{N,k}}{2}V_{N,k}dt+iV_{N,k}\sqrt{\kappa \lambda_{N,k}}dB_{N,k}(t), \quad V_{N,k}(0)=x_{N,k},
\end{equation}
where $B_{N,1},...,B_{N,N}$ are $N$ independent standard Brownian motions and $\kappa\in(0,4].$ In this case it is known that the solutions $V_{N,1},...,V_{N,N}$ exist for all $t\geq 0$ and they don't hit each other.\\

The corresponding $N$-slit Loewner equation \begin{equation}\label{multi2}\frac{d}{dt}g_t(z) = \sum_{k=1}^N \lambda_{N,k}  g_t \frac{V_{N,k}(t)+g_t(z)}{V_{N,k}(t)-g_t(z)}, \quad g_0(z)=z\in\D, \end{equation}
describes the growth of $N$ multiple SLE (=Stochastic Loewner Evolution) curves growing from $x_{N,1},...,x_{N,N}$ to $0;$ see \cite{MR2187598}, p. 1130 (where the function $Z$ has to be taken from equation (4) on p. 1138).\\

We are interested in the limit case $N\to\infty$:\\

Let $\delta_x$ be the point measure in $x$ with mass 1 and let $\mu_{N,t}=\sum_{k=1}^N \lambda_{N,k} \delta_{V_{N,k}(t)}$. Then equation \eqref{multi2} can be written as 
$$\frac{d}{dt}g_t=\int_{\R}\frac{2}{g_{t}-u}\, \mu_{N,t}(du).$$

Assume \begin{equation}\mu_{N,0} \overset{\bold w}{\longrightarrow} \mu \quad \text{as $N\to\infty$,}\end{equation}
where we denote by $\overset{\bold w}{\longrightarrow}$ weak convergence. \\
$$\text{Does the limit measure $\mu_t = \lim_{N\to\infty}\mu_{N,t}$ exist? How can it be described?}$$
\newpage
\section{Measure-valued processes}

Fix $T>0$ and let $\mathcal{P}(\R)$ be the space of probability measures on $\R$ endowed with the topology of weak convergence
(metric space due to L\'{e}vy-Prokhorov metric). We denote by $\mathcal{M}=C([0,T], \mathcal{P}(\R))$ the space of all continuous measure-valued processes on $[0,T]$ endowed with the topology of uniform convergence.\\
For every $N\in\N,$ $\mu_{N,t}$ can be regarded as a random element from $\mathcal{M}$ and we want to show that $\mu_{N,t}$ converges in distribution w.r.t. to the topology of $\mathcal{M}.$\\
This can be reduced to the convergence of stochastic \emph{real-valued} processes (see \cite{MR1176727, MR1217451, MR1440140} and \cite[p.107ff.]{MR838085}):\\
Let $(f_n)_{n\in\N}$ be a dense sequence in $C^\infty_b(\R)$ of bounded smooth functions. It is sufficient to show that for each $n\in\N$ the sequence of real-valued processes
\begin{equation}\label{eq:dist}
 \int_\R f_n(x) \, \mu_{N,t}(dx)
\end{equation}
(as a sequence of random elements from the space $C([0,T], \R)$ with uniform convergence)
converges in distribution as $N\to\infty.$\\ 


\section{The simultaneous case}

In the case $\lambda_{N,k} = \frac1{N}$ for all $k$, which we call the \emph{simultaneous} case, equation \eqref{sigma} becomes
\begin{equation}\label{sigma_simul}
 \sum_{j\not=k}\frac{2}{N} V_{N,k}(t)\frac{V_{N,j}(t)+V_{N,k}(t)}{V_{N,j}(t)-V_{N,k}(t)}dt-\frac{\kappa}{2N}V_{N,k}dt+iV_{N,k}\sqrt{\frac{\kappa}{N}}dB_{N,k}(t), \quad V_{N,k}(0)=x_{N,k}.
\end{equation}

The corresponding $N$-slit Loewner equation \begin{equation}\label{multi2}\frac{d}{dt}g_{N,t}(z) = \sum_{k=1}^N \frac1{N}  g_{N,t} \frac{V_{N,k}(t)+g_{N,t}(z)}{V_{N,k}(t)-g_{N,t}(z)}, \quad g_{N,0}(z)=z\in\D, \end{equation}
describes the growth of $N$ multiple SLE (=Stochastic Loewner Evolution) curves growing from $x_{N,1},...,x_{N,N}$ to $0;$ see \cite{MR2004294}, equations (4) and (5). Here we chose the time $t$ such that $g'_t(0)=e^{-t}.$\\

\begin{theorem}
Assume $\mu_{N,0}$ converges weakly to a measure $\mu_0.$ Then $\mu_{N,t}$ converges to the deterministic process $\mu_t$ defined as the unique solution to the equation
\begin{equation}
\label{eq:1}
\frac{d}{dt}\left(\int_{\partial\D}f(x) \,\mu_t(dx)\right) = -\int_{}\int_{\partial\D^2} (x+y)\left(x\frac{f'(x)-f'(y)}{x-y} + f'(y)\right) \,
\mu_{t}(dx) \mu_{t}(dy)
\end{equation}
for all $f\in C^\infty(\partial\D,\C).$ The solution $g_{N,t}$ converges locally uniformly to $g_t,$ the solution of the Loewner equation
\begin{equation}
\label{eq:2}
\frac{d}{dt}g_{t}(z) = \int_{\partial\D}\  g_{t} \frac{u+g_{t}(z)}{u-g_{t}(z)}\,\mu_t(du), \quad g_{N,0}(z)=z\in\D.
\end{equation}
\end{theorem}

Taking $f=\frac{2}{z-x},$ $z\in\Ha,$ and writing $M_t(z)=\int_\R \frac{2}{z-x}\,\mu_t(dx)$ leads to the Burgers equation
$$\frac{\partial}{\partial t} M_t = -2M_t\cdot \frac{\partial}{\partial z}M_t(z),$$
and the solutions to the Loewner equation \eqref{multi2} converge to the solution of 
$$ \frac{d}{dt}g_t = M_t(g_t). $$

\section{The non-simultaneous case}

Now we are interested in the \emph{non-simultaneous} case where $\lambda_{N,k}\not=\frac1{N}$ in general.

\subsection{Idea}

Besides $\mu_{N,t}$ we also need the measure $$\tilde{\mu}_{N,t}=\sum_{k=1}^N \frac{1}{N} \delta_{V_{N,k}(t)}.$$
Then $\mu_{N,t}$ is absolutely continuous with respect to $\tilde{\mu}_{N,t}.$

While $\tilde{\mu}_{N,t}$ tells us ``where'' the driving functions are concentrated at $N=\infty,$ the measure $\mu_{N,t}$ (together with $\tilde{\mu}_{N,t}$) carries the information of the $\lambda$'s at $N=\infty.$\\
We now make the following assumptions:
\begin{itemize}
\item[(a)] \begin{equation}
\mu_{N,0} \overset{{\bold w}}{\longrightarrow} \mu_{0} \quad \text{as $N\to\infty$,}
\label{eq:1}
\end{equation}
\item[(b)] \begin{equation}
\tilde{\mu}_{N,0}\overset{{\bold w}}{\longrightarrow}\tilde{\mu}_{0} \quad \text{as $N\to\infty$.}
\label{eq:2}
\end{equation}
\end{itemize}

We will need some further assumptions as the following two examples show.

\begin{example} Let us consider only odd $N\in\N$ and let $\hat{N}=(N+1)/2.$\\ 
Assume $x_{N,\hat{N}}=0,$ $\lambda_{N,\hat{N}}=\frac1{2}$, $|x_{N,\hat{N},k}|\geq 1$ for all $k\neq \hat{N}$ and all $N\in\N.$ Furthermore, we choose $x_{N,k},\lambda_{N,k}$ for $k\neq N$ such that both $\mu_{N,0}$ and $\tilde{\mu}_{N,0}$ are symmetric with respect to $0$.\\
In this case, there won't be a deterministic limit measure. 
\end{example}


\begin{example}\label{too_vari}
Let $S_N=1+\frac{N+1}{2N}$. Now we choose
$$x_{N,k}=\frac{k}{N^2} \text{\quad and \quad} \lambda_{N,k} = \frac{1}{S_N}\left(1 + \frac{k}{N}\right) \cdot \frac{1}{N}.$$
Obviously, $\mu_{N,0}, \tilde{\mu}_{N,0} \to \delta_0.$ However, for $f(x)=x$, it is easy to see that the time derivative
$\frac{d}{dt}\int_\R f(x) \, \mu_{N,0}(x)|_{t=0}$ goes to $\infty$ (with prob. 1) for $N\to\infty$ by using the calculations from below. 
\end{example}

%
%\begin{eqnarray*}&=&\sum_{j\not=k}\frac{\lambda_{N,k}^2 f'(V_{N,k}(t)) - \lambda_{N,j}^2 f'(V_{N,j}(t)) }{V_{N,k}(t)-V_{N,j}(t)} \\
%&=&
%\sum_{j\not=k}\frac{\lambda_{N,k}^2 f'(V_{N,k}(t)) - \lambda_{N,k}^2 f'(V_{N,j}(t)) + \lambda_{N,k}^2 f'(V_{N,j}(t)) - \lambda_{N,j}^2 f'(V_{N,j}(t)) }{V_{N,k}(t)-V_{N,j}(t)} \\
%&=& \sum_{j\not=k}\lambda_{N,k}^2 \cdot \frac{f'(V_{N,k}(t)) -  f'(V_{N,j}(t))  }{V_{N,k}(t)-V_{N,j}(t)} +
%\sum_{j\not=k} f'(V_{N,j}(t)) \cdot \frac{\lambda_{N,k}^2 - \lambda_{N,j}^2 }{V_{N,k}(t)-V_{N,j}(t)} \end{eqnarray*}
%
%\begin{eqnarray*}&&d(V_k-V_l)=\sum_{j\not=k}\frac{2(\lambda_{N,k}+\lambda_{N,j})}{V_{N,k}(t)-V_{N,j}(t)}dt - \sum_{j\not=l}\frac{2(\lambda_{N,l}+\lambda_{N,j})}{V_{N,l}(t)-V_{N,j}(t)}dt + \sqrt{\kappa \lambda_{N,k}}dB_{N,k}(t) -\sqrt{\kappa \lambda_{N,l}}dB_{N,l}(t), \quad V_{N,k}-V_{N,l}(0)=x_{N,k}-x_{N,l} \\
%&=& \frac{4(\lambda_{N,k}+\lambda_{N,l})}{V_{N,k}(t)-V_{N,l}(t)}dt + \sum_{j\not=k,l}\frac{2(\lambda_{N,k}+\lambda_{N,j})}{V_{N,k}(t)-V_{N,j}(t)}-\frac{2(\lambda_{N,l}+\lambda_{N,j})}{V_{N,l}(t)-V_{N,j}(t)} dt + ... \\
%&=&  \frac{4(\lambda_{N,k}+\lambda_{N,l})}{V_{N,k}(t)-V_{N,l}(t)}dt + 2\sum_{j\not=k,l}\frac{(\lambda_{N,k}+\lambda_{N,j})(V_{N,l}(t)-V_{N,j}(t)) - (\lambda_{N,l}+\lambda_{N,j})(V_{N,k}(t)-V_{N,j}(t))}{(V_{N,k}(t)-V_{N,j}(t))(V_{N,l}(t)-V_{N,j}(t))} dt + ... \\
%&=&   \frac{4(\lambda_{N,k}+\lambda_{N,l})}{V_{N,k}(t)-V_{N,l}(t)}dt + 2\sum_{j\not=k,l}\frac{(\lambda_{N,k}+\lambda_{N,j})V_{N,l}(t) - (\lambda_{N,l}+\lambda_{N,j})V_{N,k}(t) + (\lambda_{N,l}-\lambda_{N,k}) V_{N,j}(t)}{(V_{N,k}(t)-V_{N,j}(t))(V_{N,l}(t)-V_{N,j}(t))} dt +\end{eqnarray*}
%
In order to get a deterministic limit we will assume:
\begin{itemize}
\item[(c)] there exists $C>0$ such that for every $N\in\N:$
\begin{equation}\label{max}
\max_{k\in\{1,...,N\}} \lambda_{N,k} \leq \frac{C}{N}.
 \end{equation}
\end{itemize}
Because of assumption (c), also the limit measure $\mu_0$ is absolutely continuous w.r.t. $\tilde{\mu}_0.$ We require this in order to get a \emph{deterministic} limit.\\

In order to exclude a situation as in Example \ref{too_vari} we assume:
\begin{itemize}
\item[(d)] The following sum is bounded independently of $N$:
$$ \sum_{k\not=j} \left|\frac{\lambda_{N,k}^2-\lambda_{N,j}^2}{x_{N,k}-x_{N,j}}\right| \leq D. $$
\end{itemize}

{\color{orange} \begin{eqnarray*}
&& d/dt \sum_{k\not=j}\frac{\lambda_{N,k}^2-\lambda_{N,j}^2}{V_{N,k}(t)-V_{N,j}(t)} = \sum_{k\not=j}-\frac{\lambda_{N,k}^2-\lambda_{N,j}^2}{(V_{N,k}(t)-V_{N,j}(t))^2} dV/dt + \frac{\lambda_{N,k}^2-\lambda_{N,j}^2}{(V_{N,k}(t)-V_{N,j}(t))^2} dV/dt \\
&=&  \sum_{k\not=j}-\frac{\lambda_{N,k}^2-\lambda_{N,j}^2}{(V_{N,k}(t)-V_{N,j}(t))^2} \sum_{l\not=k}\frac{2(\lambda_{N,k}+\lambda_{N,l})}{V_{N,k}(t)-V_{N,l}(t)} \\
&+&  \frac{\lambda_{N,k}^2-\lambda_{N,j}^2}{(V_{N,k}(t)-V_{N,j}(t))^2} \sum_{l\not=j}\frac{2(\lambda_{N,j}+\lambda_{N,l})}{V_{N,j}(t)-V_{N,l}(t)} \\
&=& \sum_{k\not=j}\frac{\lambda_{N,k}^2-\lambda_{N,j}^2}{(V_{N,k}(t)-V_{N,j}(t))^2} (\sum_{l\not=k}\frac{-2(\lambda_{N,k}+\lambda_{N,l})}{V_{N,k}(t)-V_{N,l}(t)} +  \frac{2(\lambda_{N,j}+\lambda_{N,l})}{V_{N,j}(t)-V_{N,l}(t)}\\
&+& \frac{-2(\lambda_{N,k}+\lambda_{N,j})}{V_{N,k}(t)-V_{N,j}(t)}+ \frac{2(\lambda_{N,j}+\lambda_{N,k})}{V_{N,j}(t)-V_{N,k}(t)}) \\
&=& \sum_{k\not=j}\frac{\lambda_{N,k}^2-\lambda_{N,j}^2}{(V_{N,k}(t)-V_{N,j}(t))^2} (\sum_{l\not=k}\frac{-2(\lambda_{N,k}+\lambda_{N,l})}{V_{N,k}(t)-V_{N,l}(t)} +  \frac{2(\lambda_{N,j}+\lambda_{N,l})}{V_{N,j}(t)-V_{N,l}(t)}\\
&+&  \frac{4(\lambda_{N,j}+\lambda_{N,k})}{V_{N,k}(t)-V_{N,j}(t)}) \\
&=& \sum_{k>j}\frac{\lambda_{N,k}^2-\lambda_{N,j}^2}{(V_{N,k}(t)-V_{N,j}(t))^2} (\sum_{l\not=j,k}\frac{-2(\lambda_{N,k}+\lambda_{N,l})}{V_{N,k}(t)-V_{N,l}(t)} +  \frac{2(\lambda_{N,j}+\lambda_{N,l})}{V_{N,j}(t)-V_{N,l}(t)}\\
&+&  \frac{4(\lambda_{N,j}+\lambda_{N,k})}{V_{N,k}(t)-V_{N,j}(t)})  \\
&+& \sum_{k>j}\frac{\lambda_{N,k}^2-\lambda_{N,j}^2}{(V_{N,k}(t)-V_{N,j}(t))^2} (\sum_{l\not=j,k}\frac{2(\lambda_{N,j}+\lambda_{N,l})}{V_{N,j}(t)-V_{N,l}(t)} -  \frac{2(\lambda_{N,k}+\lambda_{N,l})}{V_{N,k}(t)-V_{N,l}(t)}\\
&+&  \frac{4(\lambda_{N,j}+\lambda_{N,k})}{V_{N,k}(t)-V_{N,j}(t)}) \\
&=& 2\sum_{k>j}\frac{\lambda_{N,k}^2-\lambda_{N,j}^2}{(V_{N,k}(t)-V_{N,j}(t))} 2(\sum_{l\not=j,k}\frac{(\lambda_{N,j}+\lambda_{N,l})(V_{N,k}(t)-V_{N,l}(t))-(\lambda_{N,k}+\lambda_{N,l})(V_{N,j}(t)-V_{N,l}(t))}{(V_{N,j}(t)-V_{N,l}(t))((V_{N,k}(t)-V_{N,j}(t)))(V_{N,k}(t)-V_{N,l}(t))}\\
&+&  \frac{4(\lambda_{N,j}+\lambda_{N,k})}{(V_{N,k}(t)-V_{N,j}(t))^2}) \\
\label{eq:}
\label{eq:}
\end{eqnarray*}}


An example for (d) would be the following situation:\\
 Define the function $\nu_N:\supp(\mu_{N,0})\to[0,1]$ by $x_{N,k}\mapsto \lambda_{N,k}N$. Assume there exists $D>0$ such that $\nu_N$ is Lipschitz-continuous with constant $D$ for every $N\in\N;$ in other words:
$$ \max_{j\not=k} \left|\frac{\lambda_{N,k}-\lambda_{N,j}}{x_{N,k}-x_{N,j}}\right| \leq \frac{D}{N}. $$ 
Then (d) is clearly satisfied. In Example \ref{too_vari}, however, this Lipschitz-condition is not satisfied (and neither is (d)) as 
$$\frac{\lambda_{N,N}-\lambda_{N,1}}{x_{N,N}-x_{N,1}} = \frac{\frac1{NS_N}(1-\frac1{N})}{\frac1{N}-\frac1{N^2}}\to \frac2{3}.$$


Sketch of a proof:
\begin{itemize}
\item[1)] Use idea \eqref{eq:dist} and compute $\int_\R f_n(x) \, \mu_{N,t}(dx)$ and $\int_\R f_n(x) \, \tilde{\mu}_{N,t}(dx)
$ by It\={o}'s formula. Use \cite{MR1700749}, Thm. 7.3, to show that, for each $n\in\N,$ the sequence $(\int_\R f_n(x) \, \mu_{N,t}(dx), \int_\R f_n(x) \, \tilde{\mu}_{N,t}(dx))$ is \emph{tight} (= there exists a converging subsequence).  
\item[2)] Derive a differential equation for the limit of a subsequence like \eqref{mckean}.
\item[3)] Show that this equation has a unique solution, i.e. there exists a unqiue (deterministic) process $(\mu_t, \tilde{\mu}_t)$ that satisfies the equation. This proves convergence.
\end{itemize}

\subsection{Tightness}

Denote by $\nu_{N,t}(x)$ the Radon-Nikod\'{y}m derivative of $\mu_{N,t}$ with respect to $\tilde{\mu}_{N,t}$ and define the measure $\alpha_{N,t}$ by
$$\alpha_{N,t} =  \nu_{N,t}(x)^2 \tilde{\mu}(dx) = \sum_{k=1}^N \lambda_{N,k}^2N \delta_{V_{N,k}(t)}.$$
Then $\alpha_{N,t}\times \tilde{\mu}_{N,t} = \sum_{j,k} \frac{\nu_k^2}{N} \frac1{N} \delta_{(V_k(t), V_j(t))} = \sum_{j,k} \lambda_{N,k}^2 \delta_{(V_k(t), V_j(t))}$.
 
%\begin{equation}\label{blue} {\color{blue}\sum_{k=1}^N  \lambda_k^2 f'(V_k(t))\sum_{j\not=k}\frac{2}{V_k(t)-V_j(t)} } = \int_{}\int_{x\not=y} \frac{2f'(x)}{x-y} \,
%(\hat{\mu}_{N,t}\times \tilde{\mu}_{N,t})(dxdy).
%\end{equation}
Let $f\in C^\infty_b(\R).$ Then It\={o}'s formula gives

\begin{eqnarray*}
 && d\left(\int_\R f(x) \, \mu_{N,t}(dx)\right) = 
d\left(\sum_{k=1}^N \frac{1}{N} f(V_{N,k}(t)) \right) \\
&=& \sum_{k=1}^N  \frac{2}{N^2}( f'(V_{N,k}(t))\sum_{j\not=k}V_{N,k}(t)\frac{V_{N,j}(t)+V_{N,k}(t)}{V_{N,j}(t)-V_{N,k}(t)}\,dt \\
&-& 
 f'(V_{N,k}(t)) \frac{\kappa}{2N^2}V_{N,k}\,dt
-V_{N,k}^2\frac{\kappa}{2N^2} f''(V_{N,k}(t)) \,dt +  f'(V_{N,k}(t))iV_{N,k}\frac{\sqrt{\kappa}}{\sqrt{N}N}\,dB_{N,k}(t) ) \\
%&=& \sum_{k=1}^N  \frac{1}{N} f'(V_{N,k}(t))\sum_{j\not=k}\frac{2(\lambda_{N,k}+\lambda_{N,j})}{V_{N,k}(t)-V_{N,j}(t)} \,dt  + \dots \\
&=& \int_{}\int_{x\not=y} 2f'(x)x\frac{y+x}{y-x} \,
(\mu_{N,t}\times \mu_{N,t})(dxdy) \,dt + \dots\\
&=& \int_{}\int_{x\not=y} f'(x)x\frac{y+x}{y-x} \,
(\mu_{N,t}\times \mu_{N,t})(dxdy) \,dt - \int_{}\int_{x\not=y} f'(y)y\frac{y+x}{y-x} \,
(\mu_{N,t}\times \mu_{N,t})(dxdy) \,dt + \dots\\
&=& \int_{}\int_{x\not=y} (x+y)\frac{f'(x)x-f'(y)y}{y-x} \,
(\mu_{N,t}\times \mu_{N,t})(dxdy) \,dt + \dots\\
&=& \int_{}\int_{x\not=y} (x+y)\frac{f'(x)x-f'(y)x}{y-x} +  (x+y)\frac{f'(y)x-f'(y)y}{y-x} \,
(\mu_{N,t}\times \mu_{N,t})(dxdy) \,dt + \dots\\
&=& \int_{}\int_{x\not=y} x(x+y)\frac{f'(x)-f'(y)}{y-x} -  (x+y)f'(y) \,
(\mu_{N,t}\times \mu_{N,t})(dxdy) \,dt + \dots\\
\end{eqnarray*} 

%\begin{eqnarray*}
 %&& d\left(\int_\R f(x) \, \alpha_{N,t}(dx)\right) = d\left(\sum_{k=1}^N \lambda_{N,k}^2 f(V_{N,k}(t)) \right) \\
%&=& \sum_{k=1}^N  \lambda_{N,k}^2\left( f'(V_{N,k}(t))\sum_{j\not=k}\frac{2(\lambda_{N,k}+\lambda_{N,j})}{V_{N,k}(t)-V_{N,j}(t)}\,dt + 
%\frac{\kappa \lambda_{N,k}}{2} f''(V_{N,k}(t)) \,dt +  f'(V_{N,k}(t))\sqrt{\kappa \lambda_{N,k}}\,dB_{N,k}(t) \right) \\
%&=& \sum_{k=1}^N  \lambda_{N,k}^2 f'(V_{N,k}(t))\sum_{j\not=k}\frac{2(\lambda_{N,k}+\lambda_{N,j})}{V_{N,k}(t)-V_{N,j}(t)} \,dt  + \dots \\
%&=& \sum_{k=1}^N  \lambda_{N,k}^2 f'(V_{N,k}(t))\sum_{j\not=k}\frac{2\lambda_{N,j}}{V_{N,k}(t)-V_{N,j}(t)} \,dt  + 
%{\color{blue}\sum_{k=1}^N  \lambda_{N,k}^2 f'(V_{N,k}(t))\sum_{j\not=k}\frac{2}{V_{N,k}(t)-V_{N,j}(t)} }\,dt \dots  \\
%&=& \int_{}\int_{x\not=y} \frac{f'(x)-f'(y)}{x-y} \, \_{N,t}(dx)\mu_{N,t}(dy) \,dt+ {\color{blue}\int_{}\int_{x\not=y} \frac{2f'(x)}{x-y} \,
%(\alpha_{N,t}\times \tilde{\mu}_{N,t})(dxdy) }\, dt\\
%&+& \frac{\kappa}{2}  \sum_{k=1}^N \lambda_{N,k}^2 f''(V_{N,k}(t)) \,dt +  \sum_{k=1}^N  \lambda_{N,k} f'(V_k(t))\sqrt{\kappa \lambda_{N,k}}\,dB_{N,k}(t).
%\end{eqnarray*} 
%We conjecture that in the limit case $N\to\infty,$ the first term will converge to $\int_{\R}\int_{\R} \frac{f'(x)-f'(y)}{x-y} \mu_t(dx)\mu_t(dy)\,dt$ and the terms involving $\kappa$ go to $0$ (because of \eqref{max}) - this would correspond to the simultaneous case.\\
%But what about the second term, where $\lambda_k^2$ appears?\\

%
%\textbf{Idea:} Write $\lambda_k=\nu_k/N.$ Furthermore, define the measure
%$$\hat{\mu}_{N,t} = \sum_{k=1}^N \frac1{N} \nu_k^2\delta_{V_k(t)}.$$
%
%(This measure is uniquely determined by $\mu_{t,N}$ and $\tilde{\mu}_{t,N}$ as $\nu_k$ is the Radon-Nikod\'{y}m derivative of $\mu_{t,N}$ with respect to $\tilde{\mu}_{t,N}.$)\\
%Then $\hat{\mu}_{N,t}\times \tilde{\mu}_{N,t} = \sum_{j,k} \frac{\nu_k^2}{N} \frac1{N} \delta_{(V_k(t), V_j(t))} = \sum_{j,k} \lambda_k^2 \delta_{(V_k(t), V_j(t))}$ and
 %
%\begin{equation}\label{blue} {\color{blue}\sum_{k=1}^N  \lambda_k^2 f'(V_k(t))\sum_{j\not=k}\frac{2}{V_k(t)-V_j(t)} } = \int_{}\int_{x\not=y} \frac{2f'(x)}{x-y} \,
%(\hat{\mu}_{N,t}\times \tilde{\mu}_{N,t})(dxdy).
%\end{equation}
%
%%Now assume that
%%\begin{equation}
%%\tilde{\mu}_{N,s}\to \tilde{\mu}_s, \quad \hat{\mu}_{N,s}\to \hat{\mu}_s, \quad \alpha_{N,s} \to \alpha_s \text{ for $N\to\infty$}.
%%\end{equation}
%
%One can compute some examples (for $t=0$) numerically.
%
%\begin{itemize}
%\item[(a)] It seems that \eqref{blue} converges to
%%$$ \int_{\R}\int_\R \frac{2f'(x)}{x-y} \, \hat{\mu}_s(dx) \tilde{\mu}_s(dy), $$
%$$ \int_{\R^2} \frac{2f'(x)}{x-y} \,(\hat{\mu}_{t}\times \tilde{\mu}_{t})(dxdy), $$
%provided the measures are nice, say absolutely continuous with respect to the Lebesgue measure.\\
%Obviously, this can't be true in general (consider the case where $\hat{\mu}_t, \tilde{\mu}_t$ are point measures).
%
%\item[(b)] Example: Assume $\lambda_k = 1/3 \cdot \frac2{N}$ for $\frac{1}{2} N$ points close to $-1$ and  $\lambda_k=2/3 \cdot \frac2{N}$ for $\frac{1}{2} N$ points close to $1$ such that $\mu_{0,N}\to \mu_0 = 1/3\cdot \delta_{-1} + 2/3\cdot\delta_1.$\\
%Then $\tilde{\mu}_{0,N}\to 1/2 \cdot \delta_{-1} + 1/2\cdot \delta_{1}$ and $\hat{\mu}_{N,0} \to  2/9\cdot \delta_{-1} + 8/9\cdot\delta_1.$\\
%From computing some examples numerically (for $t=0$) it seems that
%\begin{eqnarray*}
%{\color{blue}\sum_{k=1}^N  \lambda_k^2 f'(V_k(0))\sum_{j\not=k}\frac{2}{V_k(0)-V_j(0)} }&\to& 
 %\int_{x\not=y} \frac{2f'(x)}{x-y} \,
%(\hat{\mu}_{0}\times \tilde{\mu}_{0})(dxdy) + \int_{x=y} f''(x) \, (\hat{\mu}_{0}\times \tilde{\mu}_{0})(dxdy),
%\end{eqnarray*} 
%where the last integral is just $\frac{2}{9}\cdot \frac1{2} \cdot f''(-1) + \frac{8}{9}\cdot \frac1{2} \cdot f''(1).$
%%\item[(c)] If $\mu_0=\delta_0$, then $\hat{\mu}_0=\tilde{\mu}_0=\delta_0$ as well. In this case we should have the usual limit equation \eqref{mckean}.
%%\item[(c)] Example: Assume $\lambda_k = 1/3 \cdot \frac2{N}$ for $\frac{1}{2} N$ points close to $-1$ and  $\lambda_k=2/3 \cdot \frac2{N}$ for $\frac{1}{2} N$ points close to $1$ such that $\mu_{0,N}\to \mu_0 = 1/3\cdot \delta_{-1} + 2/3\cdot\delta_1.$\\
%%Then $\tilde{\mu}_0=1/2 \cdot \delta_{-1} + 1/2\cdot \delta_{1}$ and $\hat{\mu}_0 =  2/9\cdot \delta_{-1} + 8/9\cdot\delta_1.$\\
%%Numerically one finds for $f(z)=\frac{2}{z-x}$ (and $t=0$) that 
%%\begin{eqnarray*}
%%{\color{blue}\sum_{k=1}^N  \lambda_k^2 f'(V_k(0))\sum_{j\not=k}\frac{2}{V_k(0)-V_j(0)} }&\to& 
 %%2\cdot \left(\frac1{9}\cdot\frac{2}{(z+1)^2}\cdot \frac{1}{(-1)-1} +\frac4{9}\cdot\frac{2}{(z-1)^2}\cdot \frac{1}{1-(-1)} \right) \\
%%&+&  \left( 2\frac{1/3}{(z+1)^2} \cdot 2\frac{1/3}{(z+1)} +  2\frac{2/3}{(z-1)^2} \cdot 2\frac{2/3}{(z-1)}\right) \\
%%&=&  \int_{}\int_{x\not=y} \frac{2f'(x)}{x-y} \,
%%\hat{\mu}_{0}(dx) \tilde{\mu}_{0}(dy) + \int_\R f''(x) \, \hat{\mu}_{0}(dx).
%%\end{eqnarray*} 
%\end{itemize}
%
%Thus one may conjecture that \eqref{blue} converges to
%$$ \int_{x\not=y} \frac{2f'(x)}{x-y} \,
%(\hat{\mu}_{t}\times \tilde{\mu}_{t})(dxdy) + \int_{x=y} f''(x) \, (\hat{\mu}_{t}\times \tilde{\mu}_{t})(dxdy).$$
%All in all
\subsection{Convergent subsequence}

Assume there exist a converging subsequence of $(\mu_{N,t}, \tilde{\mu}_{N,t})$ which we call in the following again $(\mu_{N,t}, \tilde{\mu}_{N,t})$ to simplify notation. We denote by $(\mu_t, \tilde{\mu}_t)$ the limit process.\\ 

Fix $t\geq0.$ Because of assumption (c), the measure $\mu_t$ is again absolutely continuous w.r.t. $\tilde{\mu}_t$ with R.N. derivative $\nu_t(x)$ bounded by the constant $C.$ We define the measure $\alpha_t$ by 
$$\alpha_t(A) = \int_A \nu_t(x)^2 \,\tilde{\mu}_t.$$ 
Then
$$ \alpha_{N,t} \overset{\bold w}{\longrightarrow} \alpha_t. $$

Conjecture: The limit measures $\mu_t=\lim_{N\to\infty}\mu_{N,t}$ and  $\tilde{\mu}_t=\lim_{N\to\infty}\tilde{\mu}_{N,t}$ satisfy
\begin{eqnarray}&&\frac{d}{dt}\left(\int_\R f(x)\, \mu_t(dx)\right) = \\ \nonumber
&&\int_{\R}\int_{\R} \frac{f'(x)-f'(y)}{x-y} \mu_t(dx)\mu_t(dy) + \int_{x\not=y} \frac{2f'(x)}{x-y} \,
(\alpha_{t}\times \tilde{\mu}_{t})(dxdy) + \int_{x=y} f''(x) \, (\alpha_{t}\times \tilde{\mu}_{t})(dxdy).
\end{eqnarray}
%
%How can 
%%where $\hat{\mu}_s$ and $\tilde{\mu}_s$ are constructed as follows:\\
%%$\tilde{\mu}_s$ is the limit of the measure $\sum_{k=1}^N 1/N \delta_{V_k(t)}$. Denote by $p_s(x)$ the density of $\mu_s$ with respect to $\tilde{\mu}_s,$ i.e. $\mu_s(A)=\int_A p(x) \,\tilde{\mu}_s(dx).$ Then $\hat{\mu}(A)=\int_{A}p(x)^2\, \tilde{\mu}_s(dx).$\\
%
%Furthermore, if we take again $f(x)=\frac{2}{z-x}$ and let $M_t(z)=\int_\R \frac{2}{z-x} \, \mu_t(dx),$ can we obtain a ``nice'' differential equation for $M_t$, like $\frac{\partial}{\partial t}M_t(z) = -M_t(z)\cdot \frac{\partial}{\partial z}M_t(z) + A(z,M_t(z), \frac{\partial}{\partial z}M_t(z))$ ?\\
%
%
%Example: Assume $\lambda_k = 1/3 \cdot \frac1{2N}$ for $\frac{1}{2} N$ points close to $-1$ and  $\lambda_k=2/3 \cdot \frac1{2N}$ for $\frac{1}{2} N$ points close to $1$ such that $\mu_{0,N}\to 1/3\cdot \delta_{-1} + 2/3\cdot\delta_1.$\\
%
%Then $M_0^N\to M_0=\frac{1}{3}\frac{2}{z+1} + \frac{2}{3}\frac{2}{z-1}$ and numerically we obtain \begin{eqnarray*}
%\frac{\partial }{\partial t}M_0(z) &=& - M_0(z)\cdot \frac{\partial}{\partial z}M_0(z) - 4\cdot \left(-(\frac1{3})^2\frac{1}{(z+1)^2}\cdot \frac{-1}{2} -(\frac2{3})^2\frac{1}{(z-1)^2}\cdot \frac{1}{2} \right) \\
%&+&  -\left( 2\frac{-1/3}{(z+1)^2} \cdot 2\frac{1/3}{(z+1)} +  2\frac{-2/3}{(z-1)^2} \cdot 2\frac{2/3}{(z-1)}\right).
%\end{eqnarray*}

%\begin{remark}
%%Because of \eqref{max}, the measure $\mu_s$ is absolutley continuous w.r.t. $\tilde{\mu}_s:$
%%$$\mu_s(A) = \int_A p_s(x) \, \tilde{\mu}_s(dx).$$
%%This determines $\hat{\mu}_s$ by
%%$$ \hat{\mu}_s(A) = \int_A p^2_s(x) \, \tilde{\mu}_s(dx) $$
%%and thus also $\alpha_s.$\\
%
%A similar computation for $\tilde{\mu}_t$ lets us conjecture that
%\begin{eqnarray*}
 %d\left(\int_\R f(x) \, \tilde{\mu}_{t,N}(dx)\right) &=& d\left(\sum_{k=1}^N \frac{1}{N} f(V_k(t)) \right) \\
%&=& \sum_{k=1}^N  \frac{1}{N}\left( f'(V_k(t))\sum_{j\not=k}\frac{2(\lambda_k+\lambda_j)}{V_k(t)-V_j(t)}\,dt + 
%\frac{\kappa \lambda_k}{2} f''(V_k(t)) \,dt +  f'(V_k(t))\sqrt{\kappa \lambda_k}\,dB_k(t) \right) \\
%&=& \sum_{k=1}^N  \frac{1}{N} f'(V_k(t))\sum_{j\not=k}\frac{2(\lambda_k+\lambda_j)}{V_k(t)-V_j(t)} \,dt  + \dots \\
%&=& \sum_{k=1}^N  \frac{1}{N} f'(V_k(t))\sum_{j\not=k}\frac{2\lambda_j}{V_k(t)-V_j(t)} \,dt  + 
%\sum_{k=1}^N  \frac{\lambda_k}{N}f'(V_k(t))\sum_{j\not=k}\frac{2}{V_k(t)-V_j(t)} \,dt \dots \\
%&=& \int_{}\int_{x\not=y} \frac{2f'(x)}{x-y} \,
%\tilde{\mu}_{N,s}(dx) \mu_{N,s}(dy) + \int_{}\int_{x\not=y} \frac{2f'(x)}{x-y} \,
%\mu_{N,s}(dx) \tilde{\mu}_{N,s}(dy).
%\end{eqnarray*} 

\begin{eqnarray*}  &&\frac{d}{dt}\left(\int_\R f(x)\, \tilde{\mu}_t(dx)\right) =  \int_{x\not=y} \frac{2f'(x)}{x-y} \,(\tilde{\mu}_t\times\mu_t)(dxdy) + 
 \int_{x\not=y} \frac{2f'(x)}{x-y} \,(\mu_t\times\tilde{\mu}_t)(dxdy) \\
&+& \int_{x=y} f''(x) \, (\tilde{\mu}_t\times\mu_t)(dxdy) + \int_{x=y} f''(x) \, (\mu_t\times\tilde{\mu}_t)(dxdy) \\
&=&   \int_{x\not=y} \frac{2f'(x)}{x-y} \,(\tilde{\mu}_t\times \mu_t)(dxdy)+ 
 \int_{x\not=y} \frac{2f'(x)}{x-y} \,(\mu_t\times\tilde{\mu}_t)(dxdy) +  2\int_{x=y} f''(x) \, (\mu_t\times\tilde{\mu}_t)(dxdy)=\\
&=&   2\int_{\R^2} \frac{f'(x)-f'(y)}{x-y} \,(\tilde{\mu}_t\times \mu_t)(dxdy). \end{eqnarray*} 

%\end{remark}


\bibliographystyle{amsalpha}
\bibliography{bibdata} 




\end{document}
